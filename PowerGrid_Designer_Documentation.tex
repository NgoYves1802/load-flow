\documentclass[11pt,a4paper]{article}
\usepackage[utf8]{inputenc}
\usepackage[T1]{fontenc}
\usepackage[english]{babel}
\usepackage{amsmath,amssymb,amsthm}
\usepackage{graphicx}
\usepackage{float}
\usepackage{hyperref}
\usepackage{xcolor}
\usepackage{geometry}
\usepackage{fancyhdr}
\usepackage{listings}
\usepackage{booktabs}
\usepackage{subcaption}
\usepackage{enumitem}

\geometry{margin=1in}
\pagestyle{fancy}
\fancyhf{}
\fancyhead[L]{\textbf{PowerGrid Designer Documentation}}
\fancyhead[R]{\thepage}

\hypersetup{
    colorlinks=true,
    linkcolor=blue,
    filecolor=magenta,
    urlcolor=cyan,
    pdftitle={PowerGrid Designer Documentation},
    pdfpagemode=FullScreen,
}

\lstset{
    language=JavaScript,
    basicstyle=\ttfamily\footnotesize,
    keywordstyle=\color{blue},
    commentstyle=\color{green!60!black},
    stringstyle=\color{red},
    numbers=left,
    numberstyle=\tiny,
    frame=single,
    breaklines=true,
    showstringspaces=false
}

\title{PowerGrid Designer: A Web-Based Power System Analysis Tool}
\author{PowerGrid Designer Team}
\date{\today}

\begin{document}

\maketitle

\begin{abstract}
PowerGrid Designer is a comprehensive web-based application for designing, analyzing, and optimizing electrical power systems. The application combines an intuitive drag-and-drop interface with advanced genetic algorithm-based load flow calculations to provide engineers and students with a powerful tool for power system analysis. This documentation provides a complete guide to understanding the application's functionality and using it effectively for power system studies.
\end{abstract}

\tableofcontents
\newpage

\section{Introduction}

\subsection{What is PowerGrid Designer?}

PowerGrid Designer is a modern web application developed for electrical power system analysis and design. The application provides engineers, researchers, and students with an intuitive platform to:

\begin{itemize}
    \item Design electrical networks using a visual drag-and-drop interface
    \item Perform load flow analysis using genetic algorithms
    \item Analyze power system performance and optimization
    \item Export designs to industry-standard formats
    \item Study various network topologies and configurations
\end{itemize}

\subsection{Key Features}

\begin{figure}[H]
    \centering
    \includegraphics[width=\textwidth]{application_interface.png}
    \caption{Main application interface showing the design canvas and control panels}
    \label{fig:main_interface}
\end{figure}

The application offers several key features:

\begin{enumerate}
    \item \textbf{Visual Network Design}: Drag-and-drop interface for creating power system networks
    \item \textbf{Genetic Algorithm Solver}: Advanced optimization-based load flow calculations
    \item \textbf{Real-time Analysis}: Live fitness convergence monitoring and result visualization
    \item \textbf{Multiple Selection Methods}: Tournament, Roulette Wheel, Rank-based, and Stochastic Universal Sampling
    \item \textbf{Industry Standards}: Support for PSS/E RAW format export
    \item \textbf{Educational Templates}: Pre-built IEEE test systems (3-bus, 5-bus, 14-bus)
    \item \textbf{Component Library}: Comprehensive library of generation, load, and transmission components
\end{enumerate}

\section{How It Functions}

\subsection{Technical Architecture}

\subsubsection{Frontend Architecture}

The application is built using modern web technologies:

\begin{itemize}
    \item \textbf{React 19}: Component-based user interface framework
    \item \textbf{TypeScript}: Type-safe JavaScript for robust development
    \item \textbf{Vite}: Fast build tool and development server
    \item \textbf{Lucide React}: Modern icon library
    \item \textbf{Tailwind CSS}: Utility-first CSS framework
\end{itemize}

\subsubsection{Algorithm Implementation}

\begin{figure}[H]
    \centering
    \includegraphics[width=0.8\textwidth]{ga_flowchart.png}
    \caption{Genetic algorithm flow for load flow optimization}
    \label{fig:ga_flowchart}
\end{figure}

The core of the application is a genetic algorithm (GA) implementation for solving power flow problems. The GA operates on the following principles:

\paragraph{Problem Formulation}
Power flow analysis involves solving the nonlinear equations that describe power system steady-state operation:

\begin{align}
P_i &= V_i \sum_{j=1}^n V_j (G_{ij} \cos \theta_{ij} + B_{ij} \sin \theta_{ij}) \\
Q_i &= V_i \sum_{j=1}^n V_j (G_{ij} \sin \theta_{ij} - B_{ij} \cos \theta_{ij})
\end{align}

Where:
\begin{itemize}
    \item $P_i, Q_i$: Active and reactive power injections at bus $i$
    \item $V_i, \theta_i$: Voltage magnitude and angle at bus $i$
    \item $G_{ij}, B_{ij}$: Real and imaginary parts of the Y-bus matrix elements
\end{itemize}

\paragraph{Genetic Algorithm Structure}

The GA implementation includes:

\begin{enumerate}
    \item \textbf{Chromosome Representation}: Each individual represents bus voltage magnitudes and angles
    \item \textbf{Fitness Function}: Minimizes power balance violations with penalty terms
    \item \textbf{Selection Methods}: Multiple selection strategies available
    \item \textbf{Crossover and Mutation}: Genetic operators for solution evolution
    \item \textbf{Convergence Criteria}: Multiple stopping conditions
\end{enumerate}

\subsubsection{Fitness Function}

The fitness function combines multiple objectives:

\begin{equation}
f = \alpha \sum (P_{calc} - P_{spec})^2 + \beta \sum (Q_{calc} - Q_{spec})^2 + \gamma \sum losses + \text{voltage penalties}
\end{equation}

Where $\alpha, \beta, \gamma$ are user-configurable penalty weights.

\subsection{Selection Methods}

The application implements four selection methods:

\subsubsection{Tournament Selection}
Randomly selects $k$ individuals and chooses the best. Tournament size is user-configurable.

\subsubsection{Roulette Wheel Selection}
Probability of selection proportional to fitness. Inverted fitness used for minimization problems.

\subsubsection{Rank-Based Selection}
Selection based on population ranking with linear probability distribution.

\subsubsection{Stochastic Universal Sampling}
Multiple selection points ensure better population diversity.

\section{Getting Started}

\subsection{System Requirements}

\begin{itemize}
    \item Modern web browser (Chrome, Firefox, Safari, Edge)
    \item Internet connection for initial load
    \item Minimum 4GB RAM recommended
    \item Screen resolution 1280x720 or higher
\end{itemize}

\subsection{Installation and Setup}

\subsubsection{Local Development Setup}

\begin{lstlisting}[caption=Installation commands]
# Clone the repository
git clone <repository-url>
cd powergrid-designer

# Install dependencies
npm install

# Set up Gemini API key
echo "GEMINI_API_KEY=your_api_key_here" > .env.local

# Start development server
npm run dev
\end{lstlisting}

\subsubsection{Accessing the Application}

\begin{enumerate}
    \item Open a web browser
    \item Navigate to \url{http://localhost:5173} (development) or the deployed URL
    \item The application loads with an empty canvas
\end{enumerate}

\section{User Guide}

\subsection{Interface Overview}

\begin{figure}[H]
    \centering
    \includegraphics[width=\textwidth]{interface_overview.png}
    \caption{Main interface components and layout}
    \label{fig:interface_overview}
\end{figure}

The main interface consists of:

\begin{enumerate}
    \item \textbf{Header Bar}: Application title, navigation menus, undo/redo controls
    \item \textbf{Library Panel}: Component library for adding network elements
    \item \textbf{Design Canvas}: Main drawing area for network design
    \item \textbf{Control Panel}: Parameter settings and solver controls
    \item \textbf{Results Panel}: Analysis results and convergence plots
\end{enumerate}

\subsection{Creating Your First Network}

\subsubsection{Step 1: Choose a Template}

\begin{figure}[H]
    \centering
    \includegraphics[width=0.7\textwidth]{template_selection.png}
    \caption{Template selection menu with IEEE test systems}
    \label{fig:template_selection}
\end{figure}

\begin{enumerate}
    \item Click the \textbf{Import} button in the header
    \item Select \textbf{Example Templates}
    \item Choose from:
    \begin{itemize}
        \item IEEE 3-Bus System (simple radial system)
        \item IEEE 5-Bus System (meshed network)
        \item IEEE 14-Bus System (comprehensive test case)
        \item Radial Feeder (distribution system example)
    \end{itemize}
\end{enumerate}

\subsubsection{Step 2: Understand Component Types}

\begin{figure}[H]
    \centering
    \includegraphics[width=0.8\textwidth]{component_library.png}
    \caption{Component library showing different element types}
    \label{fig:component_library}
\end{figure}

The application supports three main bus types:

\paragraph{Slack Bus (Reference)}
\begin{itemize}
    \item Maintains system frequency and voltage reference
    \item Balances system power requirements
    \item Usually represents the main power source
\end{itemize}

\paragraph{PV Bus (Generator)}
\begin{itemize}
    \item Fixed active power and voltage magnitude
    \item Reactive power adjusts to maintain voltage
    \item Represents synchronous generators
\end{itemize}

\paragraph{PQ Bus (Load)}
\begin{itemize}
    \item Fixed active and reactive power consumption
    \item Voltage and angle are calculated
    \item Represents loads or infinite buses
\end{itemize}

\subsection{Building Custom Networks}

\subsubsection{Adding Components}

\begin{enumerate}
    \item Open the \textbf{Library Panel} (click the "Library" button)
    \item Select a component category (Generation, Load, Transmission)
    \item Drag components onto the canvas
    \item Connect components by clicking the "+" button on source buses and then destination buses
\end{enumerate}

\subsubsection{Component Configuration}

\begin{figure}[H]
    \centering
    \includegraphics[width=0.6\textwidth]{component_properties.png}
    \caption{Component properties panel for parameter editing}
    \label{fig:component_properties}
\end{figure}

Select any component on the canvas to edit its properties:

\begin{itemize}
    \item \textbf{Bus Properties}: Name, voltage setpoint, power ratings
    \item \textbf{Line Properties}: Resistance, reactance, charging susceptance, thermal limits
    \item \textbf{Transformer Properties}: Tap ratios, impedance values
\end{itemize}

\subsection{Running Load Flow Analysis}

\subsubsection{Configure Solver Parameters}

\begin{figure}[H]
    \centering
    \includegraphics[width=0.7\textwidth]{solver_parameters.png}
    \caption{Genetic algorithm parameter configuration panel}
    \label{fig:solver_parameters}
\end{figure}

Switch to the \textbf{Parameters} tab in the control panel:

\paragraph{Population Settings}
\begin{itemize}
    \item \textbf{Population Size}: Number of individuals (50-1000)
    \item \textbf{Max Generations}: Maximum iterations (100-2000)
\end{itemize}

\paragraph{Genetic Operators}
\begin{itemize}
    \item \textbf{Crossover Rate}: Probability of crossover (0.6-0.9)
    \item \textbf{Mutation Rate}: Probability of mutation (0.01-0.1)
\end{itemize}

\paragraph{Selection Method}
\begin{itemize}
    \item Tournament (recommended for most cases)
    \item Roulette Wheel (fitness proportional)
    \item Rank-based (ranking-based selection)
    \item Stochastic Universal (diversity preservation)
\end{itemize}

\paragraph{Penalty Weights}
\begin{itemize}
    \item Alpha: Active power violation penalty
    \item Beta: Reactive power violation penalty
    \item Gamma: Power loss penalty
\end{itemize}

\subsubsection{Execute Analysis}

\begin{enumerate}
    \item Click the \textbf{"Run Solver"} button
    \item Monitor progress in the status indicator
    \item View real-time fitness convergence in the results panel
    \item Analysis completes when convergence criteria are met
\end{enumerate}

\subsection{Analyzing Results}

\subsubsection{Understanding Output}

\begin{figure}[H]
    \centering
    \includegraphics[width=\textwidth]{results_analysis.png}
    \caption{Results panel showing bus voltages, power flows, and convergence}
    \label{fig:results_analysis}
\end{figure}

The results panel displays:

\paragraph{Performance Summary}
\begin{itemize}
    \item Final fitness value
    \item Number of generations executed
    \item Convergence status
\end{itemize}

\paragraph{Bus Results Table}
\begin{itemize}
    \item Calculated voltage magnitudes and angles
    \item Active and reactive power injections
    \item Power balance verification
\end{itemize}

\paragraph{Line Results Table}
\begin{itemize}
    \item Power flows in each direction
    \item Line losses
    \item Loading percentages
\end{itemize}

\paragraph{Convergence Plot}
Real-time visualization of fitness improvement over generations.

\subsubsection{Exporting Results}

\begin{enumerate}
    \item Click the \textbf{Export} button in the header
    \item Choose format:
    \begin{itemize}
        \item \textbf{JSON Config}: Save network design
        \item \textbf{PSS/E RAW}: Export to power system simulation software
    \end{itemize}
    \item Results are automatically downloaded to your computer
\end{enumerate}

\subsection{Advanced Features}

\subsubsection{Undo/Redo System}

\begin{itemize}
    \item \textbf{Ctrl+Z / Cmd+Z}: Undo last action
    \item \textbf{Ctrl+Y / Cmd+Y}: Redo action
    \item \textbf{Visual History}: Track changes in the status bar
\end{itemize}

\subsubsection{Canvas Navigation}

\begin{itemize}
    \item \textbf{Zoom}: Mouse wheel or zoom buttons (+/-)
    \item \textbf{Pan}: Click and drag empty canvas areas
    \item \textbf{Fit to Screen}: Double-click canvas or reset zoom button
\end{itemize}

\subsubsection{Component Management}

\begin{itemize}
    \item \textbf{Delete}: Select component and press Delete key
    \item \textbf{Move}: Drag components to new positions
    \item \textbf{Copy/Paste}: Not currently implemented (use templates)
\end{itemize}

\section{Tutorials and Examples}

\subsection{Basic Tutorial: IEEE 3-Bus System}

\begin{figure}[H]
    \centering
    \includegraphics[width=0.8\textwidth]{ieee3_tutorial.png}
    \caption{IEEE 3-bus system tutorial setup}
    \label{fig:ieee3_tutorial}
\end{figure}

This tutorial demonstrates basic power flow analysis:

\begin{enumerate}
    \item Load the IEEE 3-Bus template
    \item Examine the network topology (1 slack, 1 PV, 1 PQ bus)
    \item Configure GA parameters (use defaults for first run)
    \item Run the solver and observe convergence
    \item Analyze the results and power flows
\end{enumerate}

\subsubsection{Expected Results}
\begin{itemize}
    \item System converges within 100 generations
    \item Slack bus provides power balance
    \item PV bus maintains voltage while supplying power
    \item PQ bus voltage adjusts to meet load requirements
\end{itemize}

\subsection{Advanced Tutorial: Network Optimization}

\begin{enumerate}
    \item Start with a radial feeder system
    \item Add multiple load points
    \item Experiment with different selection methods
    \item Adjust penalty weights to optimize for different objectives
    \item Compare convergence rates and solution quality
\end{enumerate}

\subsection{Troubleshooting Common Issues}

\subsubsection{Solver Won't Converge}

\begin{itemize}
    \item \textbf{Check Network Topology}: Ensure system is properly connected
    \item \textbf{Adjust Parameters}: Increase population size or generations
    \item \textbf{Modify Constraints}: Relax voltage limits if too restrictive
    \item \textbf{Balance Penalties}: Adjust alpha/beta/gamma weights
\end{itemize}

\subsubsection{Unrealistic Results}

\begin{itemize}
    \item \textbf{Verify Input Data}: Check component ratings and connections
    \item \textbf{Stability Issues}: System may be operating outside stable region
    \item \textbf{Parameter Tuning}: Try different GA parameter combinations
\end{itemize}

\subsubsection{Performance Issues}

\begin{itemize}
    \item \textbf{Large Networks}: Reduce population size for faster computation
    \item \textbf{Browser Resources}: Close other tabs to free memory
    \item \textbf{Convergence Criteria}: Increase convergence threshold for faster stopping
\end{itemize}

\section{Technical Reference}

\subsection{Algorithm Parameters}

\begin{table}[H]
    \centering
    \caption{Recommended parameter ranges for different problem sizes}
    \label{tab:parameters}
    \begin{tabular}{@{}lccc@{}}
        \toprule
        Parameter & Small System & Medium System & Large System \\
        \midrule
        Population Size & 50-100 & 100-200 & 200-500 \\
        Max Generations & 200-500 & 500-1000 & 1000-2000 \\
        Crossover Rate & 0.7-0.9 & 0.7-0.9 & 0.6-0.8 \\
        Mutation Rate & 0.05-0.1 & 0.03-0.08 & 0.01-0.05 \\
        Tournament Size & 3-5 & 5-7 & 7-10 \\
        \bottomrule
    \end{tabular}
\end{table}

\subsection{Mathematical Formulation}

\subsubsection{Y-Bus Matrix Construction}

The admittance matrix is constructed from line and transformer parameters:

\begin{equation}
Y_{ij} = -\frac{1}{Z_{ij}} + j\frac{B_{ij}}{2}
\end{equation}

\subsubsection{Power Flow Equations}

For each bus $i$:

\begin{align}
P_i - jQ_i &= V_i^* I_i \\
I_i &= \sum_{j=1}^n Y_{ij} V_j \\
P_i - jQ_i &= V_i^* \sum_{j=1}^n Y_{ij} V_j
\end{align}

\subsubsection{Fitness Evaluation}

The fitness function evaluates solution quality:

\begin{equation}
f(\mathbf{x}) = \sum_{i=1}^{n_{bus}} \left[ \alpha(P_{i,calc} - P_{i,spec})^2 + \beta(Q_{i,calc} - Q_{i,spec})^2 \right] + \gamma P_{loss} + V_{penalties}
\end{equation}

\subsection{File Formats}

\subsubsection{JSON Network Format}

\begin{lstlisting}[language=JSON, caption=Network design JSON structure]
{
  "nodes": [
    {
      "id": "bus-1",
      "name": "Slack Bus",
      "type": "SLACK",
      "voltage": 1.05,
      "pGen": 100,
      "qGen": 20,
      "x": 400,
      "y": 200
    }
  ],
  "edges": [
    {
      "id": "line-1",
      "from": "bus-1",
      "to": "bus-2",
      "resistance": 0.02,
      "reactance": 0.06,
      "susceptance": 0.03
    }
  ]
}
\end{lstlisting}

\subsubsection{PSS/E RAW Format}

The application exports to PSS/E Version 33 RAW format for compatibility with power system analysis software.

\section{Conclusion}

PowerGrid Designer represents a significant advancement in power system education and analysis tools. By combining intuitive visual design with sophisticated computational methods, the application makes complex power system analysis accessible to a wide range of users.

\subsection{Future Developments}

Planned enhancements include:

\begin{itemize}
    \item Additional optimization algorithms (PSO, DE, etc.)
    \item Dynamic analysis capabilities
    \item Integration with external solvers
    \item Enhanced visualization features
    \item Collaborative design features
\end{itemize}

\subsection{Support and Resources}

\begin{itemize}
    \item \textbf{GitHub Repository}: Source code and issue tracking
    \item \textbf{Documentation}: Comprehensive user guides and API references
    \item \textbf{Community}: Discussion forums and user groups
    \item \textbf{Education}: Tutorial videos and example problems
\end{itemize}

\subsection{Acknowledgments}

This application was developed to advance power system education and research. Special thanks to the open-source community and academic contributors who have made this work possible.

\end{document}
